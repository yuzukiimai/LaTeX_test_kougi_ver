\documentclass{jarticle}

\title{数値解析学1の練習}
\author{今井悠月}
\date{\today}

\begin{document}

\maketitle

\section{はじめに}

この文書は、数値解析学1の授業で \LaTeX の使い方を学ぶために作っているサンプル文書です。
たとえば、数式を入力するとこんな具合になります。

\begin{equation}
x(t) = \frac{a_0}{2} + \sum_{n=1}^{\infty} \left ( a_n \cos n\omega t + b_n \sin n\omega t \right )
\label{eqn:fourier}
\end{equation}

\section{次の章}

ここは次の章の文章です。先ほどの式(\ref{eqn:fourier})では...

\end{document}
